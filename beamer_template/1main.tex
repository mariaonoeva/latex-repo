%----------------------------------------------------------------------------------------
%	PACKAGES AND THEMES
% !TeX program = xelatex
%----------------------------------------------------------------------------------------
\documentclass[xcolor=dvipsnames]{beamer}
\usetheme{SimplePlus}
\usepackage{polyglossia}
\usepackage{amssymb}
\usepackage{amsmath} %math fractions
\usepackage{hyperref}
\usepackage{graphicx} % Allows including images
\usepackage{booktabs} % Allows the use of \toprule, \midrule and \bottomrule in tables
\usepackage{linguex,cgloss}
\usepackage{natbib}
\usepackage{soul} % to highlight text 
%\usepackage{color} % to color text 
\usepackage{multicol} % pretty clear i guess :]
%\usepackage[pdftex]{graphicx}
\usepackage{array} % for tables, do not remove 
\newcolumntype{Y}{>{\centering\arraybackslash}X} %for table collumns
\usepackage{tabularx}
%\usepackage[table]{xcolor}
\usepackage{xspace}
\usepackage[export]{adjustbox}% for two pics being side by side
\bibliographystyle{sp-r}
\usepackage{libertinus} % the best font evar! 
\usepackage[T1]{fontenc}
\usepackage{stmaryrd}
\usepackage{qtree}
% \usepackage{csquotes} % for quotes
% for circuled numbers
\usepackage{amsfonts}
%\usepackage{tikz, subfigure}
\usepackage{tikz}
\usepackage{subcaption} % intead of outdated subfigure
\newcommand*\circled[1]{\tikz[baseline=(char.base)]{
            \node[shape=circle,draw,inner sep=2pt] (char) {#1};}}
\renewcommand*\familydefault{\sfdefault} %% Only if the base font of the document is to be sans serif
% macros for citations:

% for inqusitive semantics
% \tikzstyle{index on}=[inner sep=2pt, white, circle, fill=black]
% \tikzstyle{index off}=[inner sep=2pt, black, circle, draw]
% \tikzstyle{index gray}=[inner sep=2pt, black, circle, fill=lightgray]
% \tikzstyle{opaque}=[fill=gray,fill opacity=.1]
% \tikzstyle{counter}=[densely dashed] 

%removing capture
\usepackage[font=scriptsize]{caption}
%\captionsetup{labelformat=empty,labelsep=none}

\newcommand{\citeposst}[1]{\citeauthor{#1}'s (\citeyear{#1})} % produces Chomsky's (1995)
\newcommand{\citeposstpg}[2]{\citeauthor{#1}'s (\citeyear[#2]{#1})} % produces Chomsky's (1995: page)
\newcommand{\citepossalt}[1]{\citeauthor{#1}'s \citeyear{#1}} % produces Chomsky's 1995
\newcommand{\citepossaltpg}[2]{\citeauthor{#1}'s \citeyear[#2]{#1}} % produces Chomsky's 1995: page

% for cgloss.sty and sans serif (without the fisrt two lines are serif)
\renewcommand{\eachwordone}{\textsf}
\renewcommand{\eachwordtwo}{\textsf}

% for commenting in the examples 
%\setlength{\Exlabelsep}{0.3em}
\newcommand{\rcommentg}[1]{\hfill\raisebox{1.9\baselineskip}[0pt][0pt]{#1}}
\newcommand{\rcommentgg}[1]{\hfill\raisebox{1.1\baselineskip}[0pt][0pt]{#1}}

% for checkmarks in examples
\newcommand{\jdg}[1]{\makebox[0pt][r]{\normalfont#1\ignorespaces}}



%\renewcommand{\bibliographytypesize}{\scriptsize}
%----------------------------------------------------------------------------------------
%	TITLE PAGE
%----------------------------------------------------------------------------------------

% \AtBeginSection[]{
%   \begin{frame}
%   \vfill
%   \centering
%   \begin{beamercolorbox}[sep=8pt,center,shadow=true,rounded=true]{title}
%     \usebeamerfont{title}\insertsectionhead\par%
%   \end{beamercolorbox}
%   \vfill
%   \end{frame}
% }

\title[SinFonIJA16]{Negation in Russian \\ polar questions} % The short title appears at the bottom of every slide, the full title is only on the title page
\subtitle{Syntactic and semantic/pragmatic aspects}

\author{Maria Onoeva \& Radek Šimík}
\institute[CUNI]{onoevam@ff.cuni.cz, radek.simik@ff.cuni.cz} % Your institution as it will appear on the bottom of every slide, may be shorthand to save space
% {
%     Department of Computer Science and Information Engineering \\
%     National Taiwan University % Your institution for the title page
% }

\date{September 21-23, 2023 \\ SinFonIJa 16 MUNI Brno} % Date, can be changed to a custom date
\titlegraphic{\includegraphics[width=.4\textwidth]{pics/charles-university-logo.png}}


%----------------------------------------------------------------------------------------
%	PRESENTATION SLIDES
%----------------------------------------------------------------------------------------


\begin{document}
\setbeamercovered{invisible}
\setcitestyle{notesep={: },yysep={, },aysep={}}

\begin{frame}[plain]
    % Print the title page as the first slide
    \titlepage
\end{frame}

\begin{frame}{Overview}
%     % Throughout your presentation, if you choose to use \section{} and \subsection{} commands, these will automatically be printed on this slide as an overview of your presentation
     \tableofcontents
\end{frame}

\section{Intro}
\begin{frame}{Russian PQs}
    \begin{footnotesize}
        \cite{Restan1972,King1994, Satunovskij2005, Esipova2023}
    \end{footnotesize}
    
    \ex. \label{ex:pos} \ag. Kupil \textcolor<2->{purple}{\textbf<2->{li}} Maks v magazine xleb? \\ 
    bought \textsc{li} Max in shop bread \\\\
    `Did Max buy bread in the shop?' \rcommentg{V1 \textit{li}} \\
    \bg. Maks \textcolor<3->{purple}{\textbf<3->{kupil}}$_{L+H*}$ v magazine xleb? \\
    Max bought in shop bread \\\\
    `Did Max buy bread in the shop?' \rcommentg{intonation $=$ V2} \\
    \par
    \pause
    \onslide<4>{
   
    \ex. \label{ex:neg} \ag. \textcolor{RoyalBlue}{Ne} 
    \textcolor{RoyalBlue}{kupil} \textcolor{RoyalBlue}{li} Maks v magazine xleb? \\
    not bought \textsc{li} Max in shop bread \\\\
    `Didn't Max buy bread in the shop?' \rcommentg{\textcolor{RoyalBlue}{\textsc{highNPQs}}}\\
     \bg. Maks \textcolor{WildStrawberry}{ne} \textcolor{WildStrawberry}{kupil}
     v magazine xleb? \\
    Max not bought in shop bread \\\\
    `Did Max not buy bread in the shop?' \rcommentg{\textcolor{WildStrawberry}{\textsc{lowNPQs}}}\\
    \par }

\end{frame}

\begin{frame}{Research questions}

    \begin{itemize}
        \item[\circled{1}] What is the syntactic-semantic status of the negation depending 
        on the question type? \\
        \begin{scriptsize}
            \cite{Brown1995, Abels2005, Zanon2023}
        \end{scriptsize}
        \begin{itemize}
            \item V1 \textit{li}: \textcolor{RoyalBlue}{\textsc{high}} negation corresponds 
            \textcolor{Plum}{\textsc{outer}} negation (no NCIs)
            \item V2: \textcolor{WildStrawberry}{\textsc{low}} negation corresponds to both 
            \textcolor{Plum}{\textsc{outer}} and \textcolor{Orange}{\textsc{inner}} negation

            \onslide<4->{\item[$\rightarrow$] confirmed}
        \end{itemize}\pause
        \item[\circled{2}] How does negation interact with evidential bias? 
        \begin{scriptsize} \\
            \cite{buring-gunlogson00, Sudo2013, roelofsen2013positive, AnderBois2019}; a.o.
        \end{scriptsize}
        \begin{itemize}
            \item \textcolor{WildStrawberry}{\textsc{low}} negation is linked to negative evidence 
            \item \textcolor{RoyalBlue}{\textsc{high}}  negation is not felicitous with positive evidence 
            % \begin{itemize}
            %     \item negative \ding{51}, neutral \ding{51}
            % \end{itemize} 
            \onslide<5->{\item[$\rightarrow$] Russian NPQs are never more natural in negative contexts}
        \end{itemize} \pause 
        \item[\circled{3}] How does the particle \textit{razve} correlate with evidential bias? \\
        \begin{scriptsize}
            \cite{Repptoappear, Korotkovatoappear}
        \end{scriptsize}
        \begin{itemize}
            \item evidence for the prejacent and epistemic bias against it
            \onslide<6->{\item[$\rightarrow$] confirmed: evidential bias affects naturalness}
        \end{itemize}
    \end{itemize}
    
\end{frame}

% \begin{frame}{Research questions}
%     \begin{itemize}
%         \item[\circled{1}] What is the syntactic-semantic status of the negation depending on the question type? 
%         \begin{itemize}
%             \item \textcolor{RoyalBlue}{\textsc{highNPQs}} ($=$ negative V1 \textit{li}) 
%             $\times$ \textcolor{WildStrawberry}{\textsc{lowNPQs}} ($=$ negative V2)
%         \end{itemize}
%         \vspace{1em}
%         \item[\circled{2}] How does negation interact with evidential bias? 
%         \begin{itemize}
%             \item similar to English or not
%         \end{itemize}
%         \vspace{1em}
%         \item[\circled{3}] How does the particle \textit{razve} correlate with evidential bias?  

%     \end{itemize}
% \end{frame}

% \begin{frame}{Research questions}
%     \begin{itemize}
%         \item[\circled{2}] How does negation interact with evidential bias? \\
%         \begin{scriptsize}
%             \cite{buring-gunlogson00, Sudo2013, roelofsen2013positive, AnderBois2019}; a.o.
%         \end{scriptsize}
%         \begin{itemize}
%             \item \textcolor{WildStrawberry}{\textsc{low}} negation is linked to negative evidence 
%             \item \textcolor{RoyalBlue}{\textsc{high}}  negation is not felicitous with positive evidence 
%             % \begin{itemize}
%             %     \item negative \ding{51}, neutral \ding{51}
%             % \end{itemize}
%             \item Russian negative PQs are never more natural in negative contexts 
%         \end{itemize}
%     \end{itemize}
% \end{frame}

% \begin{frame}{Research questions}
%     \begin{itemize}
%         \item[\circled{3}] How does the particle \textit{razve} correlate with evidential bias? \\
%         \begin{scriptsize}
%             \cite{Repptoappear, Korotkovatoappear}
%         \end{scriptsize}
%         \begin{itemize}
%             \item evidence for the prejacent and epistemic bias against it
%         \end{itemize}
%     \end{itemize}
% \end{frame}

\section{Background}
\begin{frame}{Negation in PQs}
    \onslide<1->{
    \begin{large} {\textcolor{RoyalBlue}{\textsc{high}} $\times$ \textcolor{WildStrawberry}{\textsc{low}}}: 
    \end{large} syntactic distinction 
    \begin{small}
        \ex. \a. \textcolor{RoyalBlue}{Didn't} Sasha come to the party? \label{ex:high}
        \b. Did Sasha \textcolor{WildStrawberry}{not} come to the party? \label{ex:low}
        \par
    \end{small}
    }

    \onslide<2->{
        \begin{large}\textcolor{Plum}{\textsc{outer}} $\times$ \textcolor{Orange}{\textsc{inner}}: 
        \end{large} semantic distinction \\
        \begin{scriptsize}\cite{Ladd1981, buring-gunlogson00, romero-han04}\end{scriptsize}
        \begin{small}
            \ex. \label{ex:out-neg} \a.[Ad:] We need a person with a spouse for the experiment. 
            \b.[Sp:] \textcolor<4>{RoyalBlue}{Isn't} Natasha married? 
            \hfill \textcolor{Plum}{\textsc{outer}} $=$ \textcolor{Plum}{checking $p$}
            \par
        \end{small}}
    
        \onslide<3->{
            \begin{small}
                \ex. \label{ex:inn-neg} \a.[Ad:] Natasha is going out for a date tonight with a new partner.    
                \b.[Sp:] Is Natasha \textcolor<4>{WildStrawberry}{not} married? 
                \hfill \textcolor{Orange}{\textsc{inner}} $=$ \textcolor{Orange}{checking $\neg p$}
                \par
            \end{small}}

\end{frame}

\begin{frame}{Negation in PQs}
    polarity items \textit{too} and \textit{either} to disambiguate readings in (American) English 
    \begin{columns}
        \begin{column}{0.5\textwidth}
            \begin{itemize}
                \item \textcolor{RoyalBlue}{\textsc{high}} $\rightarrow$ 
                \textcolor{Plum}{\textsc{outer}} or \textcolor{Orange}{\textsc{inner}}
                \item \textcolor{WildStrawberry}{\textsc{low}} $\rightarrow$ \textcolor{Orange}{\textsc{inner}} \\
                \begin{scriptsize}
                    e.g. \cite{romero-han04}           
                \end{scriptsize}
            \end{itemize}
                \begin{small}
                    \ex. \a. \textcolor{RoyalBlue}{Isn't} Jane coming \textcolor{Plum}{too}? 
                    \b. \textcolor{RoyalBlue}{Isn't} Jane coming \textcolor{Orange}{either}?
                    \b. *Is Jane \textcolor{WildStrawberry}{not} coming \textcolor{Plum}{too}? 
                    \b. Is Jane \textcolor{WildStrawberry}{not} coming \textcolor{Orange}{either}? 
                    \par
                \end{small}
        \end{column}

        \begin{column}{0.5\textwidth}
            \begin{itemize}
                \item \textcolor{RoyalBlue}{\textsc{high}} $\rightarrow$ \textcolor{Plum}{\textsc{outer}}
                \item \textcolor{WildStrawberry}{\textsc{low}} $\rightarrow$ \textcolor{Orange}{\textsc{inner}} \\
                \begin{scriptsize}
                    \cite{AnderBois2019, Goodhue2022a}           
                \end{scriptsize}
            \end{itemize}
            \begin{small}
                \ex. \a. \textcolor{RoyalBlue}{Isn't} Jane coming \textcolor{Plum}{too}? 
                \b. *\textcolor{RoyalBlue}{Isn't} Jane coming \textcolor{Orange}{either}?
                \b. *Is Jane \textcolor{WildStrawberry}{not} coming \textcolor{Plum}{too}? 
                \b. Is Jane \textcolor{WildStrawberry}{not} coming \textcolor{Orange}{either}? 
                \par
            \end{small}
        \end{column}
    \end{columns}

\end{frame}

\begin{frame}{Negation in PQs: bias}
    \begin{large}\textsc{bias}\end{large}: the questioner's inclination towards one answer \\
    \begin{scriptsize}
        \cite{buring-gunlogson00, Sudo2013, Gaertner2017} 
    \end{scriptsize}
    \begin{itemize}
        \item[$\rightarrow$] \textsc{epistemic}: private questioner beliefs, knowledge, hopes, etc.
        \item[$\rightarrow$] \textsc{evidential}: contextual cues available to all interlocutors 
    \end{itemize}
    \pause 
    \begin{itemize}
        \item positive (for $p$) , negative (for $\neg p$), neutral 
        \item non-truth-conditional aspect of PQs meaning
    \end{itemize} 
    \pause
    \begin{small}
        \ex. \a. \textcolor{RoyalBlue}{Isn't} Natasha married? \hfill \textcolor{RoyalBlue}{\textsc{highNPQs}}
            \a. evidential: negative or neutral
            \b. epistemic: she is married -- positive
            \z. 
            \vspace{.5em}
        \b. Is Natasha \textcolor{WildStrawberry}{not} married? \hfill \textcolor{WildStrawberry}{\textsc{lowNPQs}}
            \a. evidential: she is not married -- negative
            \b. epistemic: positive
        \par 
    \end{small}
  
\end{frame}

\begin{frame}{Bias in Russian PQs: \textit{razve}}
    \begin{scriptsize}
        \cite{Repptoappear, Korotkovatoappear}
    \end{scriptsize}
    \ex.\ag. Razve Egor uexal v Venu?  \\
    \textsc{razve} Egor left in Vienna \\\\
    `Did Egor go to Vienna? (I believe he didn't.)' \\  
    \a. evidential: he is in Vienna -- positive
    \b. epistemic: he is not in Vienna -- negative  
    \z. 
    \vspace{0.5em}
    \pause
    \bg. Razve Egor ne uexal v Venu?  \\
    \textsc{razve} Egor not left in Vienna \\\\
    `Didn't Egor go to Vienna?' \\  
    \a. evidential: he is not in Vienna -- negative
    \b. epistemic: he is in Vienna -- positive  
    \z. 
    \par

\end{frame}

\begin{frame}{Negation and indefinites in Russian}
    \begin{scriptsize}
        \cite{Brown1999, Haspelmath2001, Geist2008, Marti2019, Kuhn2021}
    \end{scriptsize}
    \begin{itemize}
        % \item propositional negation attached to a finite verb
        \item strict negative concord lanaguage 
        \begin{itemize}
            \item negative concord items (NCIs): e.g. \textit{nikakoj} `no-which'
        \end{itemize}
        \item narrow scope non-specific indefinites -- wh-\textit{nibud'} 
        \begin{itemize}
            \item appears in the scope of some operators
        \end{itemize}
    \end{itemize}
    \begin{small}
        \ex.\ag. Nastja pročitala \{*kakuju-nibud' / *nikakuju\} knigu. \\
        Nastja read which.\textsc{nibud'}  {} which.\textsc{nci} book \\\\
        `Nastja read a book.' \\  
        \vspace{0.5em}
        \bg. Nastja ne pročitala \{*kakuju-nibud' / nikakuju\} knigu. \\
        Nastja not read which.\textsc{nibud'}  {} which.\textsc{nci} book \\\\
        `Nastja didn't read a book.' \\ 
        \par
        \pause
        \vspace{-0.5em}
        \exg. Nastja xočet pročitat' kakuju-nibud' knigu. \\ 
        Nastja wants to-read  which.\textsc{nibud'} book \\\\
        `Nastja wants to read any book.' \\
        \par
    \end{small}
\end{frame}

\begin{frame}{Negation and indefinites in Russian PQs}
    \begin{scriptsize}
        \cite{Brown1995, Abels2005, Zanon2023}
    \end{scriptsize}

    \begin{small}
        \ex.\ag. Nastja pročitala \{kakuju-nibud' / *nikakuju\} knigu? \\
        Nastja read which.\textsc{nibud'}  {} which.\textsc{nci} book \\\\
        `Did Nastja read any book?' \rcommentg{V2} \\
        \vspace{.5em}
        \bg. Nastja \textcolor{WildStrawberry}{ne} \textcolor{WildStrawberry}{pročitala} \{kakuju-nibud' / nikakuju\} knigu? \\
        Nastja not read which.\textsc{nibud'}  {} which.\textsc{nci} book \\\\
        `Did Nastja not read any book?' \rcommentg{\textcolor{WildStrawberry}{\textsc{low}}} \\
        \par
    \end{small}
    \pause 
    %\textcolor{RoyalBlue}{\textsc{highNPQs}} \\
    \begin{small}
        \ex. \ag. Pročitala li Nastja \{kakuju-nibud' / *nikakuju\} knigu? \\
        read \textsc{li} Nastja which.\textsc{nibud'}  {} which.\textsc{nci} book \\\\
        `Did Nastja read any book?' \rcommentg{V1 \textit{li}} \\
        \vspace{.5em}
        \bg. \textcolor{RoyalBlue}{Ne} \textcolor{RoyalBlue}{pročitala} \textcolor{RoyalBlue}{li} Nastja \{kakuju-nibud' / *nikakuju\} knigu? \\
        not read \textsc{li} Nastja which.\textsc{nibud'}  {} which.\textsc{nci} book \\\\
        `Did Nastja read any book?' \rcommentg{\textcolor{RoyalBlue}{\textsc{high}}}  \\ 
        \par  
    \end{small}
\end{frame}

\begin{frame}{Negation in Russian PQs: predictions}
    \begin{large}
        \textcolor{RoyalBlue}{\textsc{highNPQs}} \\
    \end{large}
    \begin{itemize}
        \item \cite{Brown1995, Abels2005}: negation is too high for NCIs licensing $\rightarrow$
         \textcolor{Plum}{\textsc{outer}} 
         \begin{itemize}
            \item \cite{Zanon2023}: polarity items \textit{eščë} `still, yet' and \textit{uže} `already' 
            are available $\rightarrow$ \textcolor{Plum}{\textsc{outer}} and \textcolor{Orange}{\textsc{inner}}
         \end{itemize}
        \item evidential bias: 
        \begin{itemize}
            \item \textcolor{Plum}{\textsc{outer}}: checking $p$ $\rightarrow$ 
            neutral or positive 
        \end{itemize}
    
    \end{itemize}

    \begin{large}
        \textcolor{WildStrawberry}{\textsc{lowNPQs}} \\
    \end{large}

    \begin{itemize}
        \item \textcolor{Plum}{\textsc{outer}} and \textcolor{Orange}{\textsc{inner}}
    \end{itemize}
        \begin{itemize}
            \item evidential bias: 
            \begin{itemize}
                \item \textcolor{Plum}{\textsc{outer}}: checking $p$ $\rightarrow$ 
                neutral or positive 
                \item \textcolor{Orange}{\textsc{inner}}: checking $\neg p$ $\rightarrow$ 
                negative
            \end{itemize}
        \end{itemize}
\end{frame}

\section{Experiment}

\begin{frame}{Participants and method}
    \begin{large} naturalness judgment task \end{large} (replication of \citealp{Stankova2023})
    \begin{itemize}
        \item rate PQs in context 
        \item Likert scale from 1 `completely unnatural' to 7 `completely natural'
        \item run on L-Rex \citep{Lrex}
        \item 68 participants found online, not paid  
    \end{itemize}

    % Experiment 1 and 2: 
    % \begin{itemize}
    %     \item 2 $\times$ 2 $\times$ 2 
    %     \item \textsc{context}, \textsc{strategy}, \textsc{indefinite}/\textsc{polarity}
    % \end{itemize}
\end{frame}

\begin{frame}{Design and materials}
    82 items in total 
    \begin{itemize}
        \item 32 items -- NPQs experiment (main)
        \item 50 items -- secondary filler experiments
        \begin{itemize}
            \item PPQs vs NPQs -- 8 items 
        \item \textit{razve}-PQs -- 8 items 
        \end{itemize}
        \item within-items and within-subjects manipulation
        \item written stimuli distributed on lists (Latin square)
    \end{itemize}
    \begin{table}[h]
        \centering
        \begin{tabular}{lcl}
            \toprule
            NPQs main & 2 $\times$ 2 $\times$ 2 & \textsc{context, strategy, indefinite} \\
            PPQs vs NPQs & 2 $\times$ 2 $\times$ 2 & \textsc{context, strategy, polarity} \\
            \textit{razve}-PQs & 3 $\times$ 2 & \textsc{context, polarity} \\
            \bottomrule
        \end{tabular}
        \caption{Experiments to report}
        \label{tab:ex1_main}
    \end{table}


\end{frame}

\begin{frame}{Design and materials}
    \textsc{context}: evidential bias manipulation 
    \begin{itemize}
        \item neutral (A): no implication of $p$ or $\neg p$
        \item negative (A'): context implies $\neg p$ 
    \end{itemize}
    \begin{footnotesize}
    \ex. \textcolor{blue}{\textbf{Neutral}} \\
        \ag.[A:] U Kiry na dače est' teplica, \textcolor{blue}{kotoruju} \textcolor{blue}{ej} 
        \textcolor{blue}{sobrali} \textcolor{blue}{v} \textcolor{blue}{prošlom} \textcolor{blue}{godu.}  \\
        at Kira on dacha is greenhouse which her built in last year \\\\
        `Kira has a greenhouse at her dacha \textcolor{blue}{which was built last year.}' \\
        \z. \textcolor{purple}{\textbf{Negative}} \\
        \ag.[A':] U Kiry na dače est' teplica, \textcolor{purple}{v} \textcolor{purple}{kotoroj} 
        \textcolor{purple}{ona} \textcolor{purple}{vyraščivaet} \textcolor{purple}{cvety.}  \\
        at Kira on dacha is greenhouse in which she grows flowers \\\\
        `Kira has a greenhouse in her dacha \textcolor{purple}{where she grows flowers.}' \\
        \z. \textbf{Question}: to be rated from 1 to 7 \\
        \a.[B:] Doesn't Kira grow some vegetables there?
        \par 
    \end{footnotesize}
    \pause 
    \begin{small}
        \begin{itemize}
            \item additionaly for the secondary experiments -- positive (context implies $p$)
        \end{itemize}
    \end{small}
\end{frame}

\begin{frame}{Design and materials}
    \textsc{strategy}: \textcolor{RoyalBlue}{\textsc{highNPQs}} and 
    \textcolor{WildStrawberry}{\textsc{lowNPQs}} manipulation

    \vspace{1em}
    \textsc{indefinite}: \textcolor{Orange}{NCIs $\approx$ \textsc{inner}} and 
    \textcolor{Plum}{wh-\textit{nibud'} $\approx$ \textsc{outer}} manipulation

    \vspace{1em}
\begin{footnotesize}
    \ex. \ag.[B:] \textcolor{RoyalBlue}{Ne} \textcolor{RoyalBlue}{posadila} \textcolor{RoyalBlue}{li}
     tuda Kira \{\textcolor{Orange}{nikakie} / \textcolor{Plum}{kakije-nibud'}\} ovošči? \\
    not planted \textsc{li} there Kira which.\textsc{nci} {} which.\textsc{nibud'}  vegetables \\\\
    `Didn't Kira plant there any/some vegetables?' \\
    \z.
    \ag.[B':] Kira \textcolor{WildStrawberry}{ne} \textcolor{WildStrawberry}{posadila} tuda
    \{\textcolor{Orange}{nikakie} / \textcolor{Plum}{kakije-nibud'}\} ovošči? \\
    Kira not planted there which.\textsc{nci} {} which.\textsc{nibud'} vegetables \\\\
    `Did Kira not plant there any/some vegetables?' 
    \par 
\end{footnotesize}
\pause 
\begin{small}
    \begin{itemize}
        \item additionaly for the secondary experiments -- \textsc{polarity} (PPQs/NPQs)
    \end{itemize}
\end{small}

\end{frame}

\begin{frame}{Design and materials}
    \begin{figure}[h]
        \includegraphics[width=\textwidth]{pics/lrex-ex.pdf}
        \caption{Item example from L-Rex}
    \end{figure}
\end{frame}

\section{Results and discussion} 

\begin{frame}{Results: NPQs}
    \begin{columns}
        \begin{column}{0.5\textwidth}
            \begin{figure}
                \centering
                \includegraphics[width=\textwidth]{pics/e1_main_aligned.eps}
                \caption{Raw; horizontal line $=$ medians} 
                \label{fig:main1}
            \end{figure}
        \end{column}

        \begin{column}{0.5\textwidth}
            \begin{figure}
                \centering
                \includegraphics[width=\textwidth]{pics/e1_interact_aligned.eps}
                \caption{NPQs means} 
                \label{fig:inter1}
            \end{figure}
        \end{column}
    \end{columns}
\end{frame}

\begin{frame}{Results: NPQs}
    \begin{columns} \onslide<1->{\begin{column}{0.5\textwidth}
            \texttt{clmm} by \cite{ordinal2023}
            \begin{itemize}
                \item main effect of all factors} % \\ \begin{scriptsize} ($p$s $< .001$) 
                %\end{scriptsize} 
                \onslide<2->{\item interaction between \textsc{strategy} and \textsc{indefinite} 
                \begin{scriptsize} ($z = 10.046, p < .001$) \end{scriptsize}
                \begin{itemize}
                    \item wh-\textit{nibud'}: fine among all
                    \item NCIs: worse in general but much more in V1
                \end{itemize}}
                 
                \onslide<3->{\item interaction between \textsc{strategy} and \textsc{context} \\
                \begin{scriptsize} ($z = 2.855, p = .004$) \end{scriptsize}
                \begin{itemize}
                    \item neutral context better in general but much more in V1
                \end{itemize}
            \end{itemize}}
        \end{column}
    \onslide<1->{\begin{column}{0.5\textwidth}
            \begin{figure}
                \centering
                \includegraphics[width=\textwidth]{pics/e1_interact_aligned.eps}
                \caption{NPQs means}
                \label{fig:inter26}
            \end{figure}
        \end{column}}
    \end{columns}
\end{frame}

\begin{frame}{Discussion: NPQs}
    \begin{columns}
        \begin{column}{0.5\textwidth}
            \begin{itemize}
                \item[\circled{1}] What is the syntactic-\\semantic 
                status of the negation depending on the question type?
            \end{itemize}
            \vspace{1em}
            \textsc{strategy}/\textsc{indefinite} interaction
            \begin{itemize}
                \begin{small}
                    %\item NCIs are worse in general but more in V1 \textit{li}
                    \item NCIs in V1 \textit{li} ( $=$ \textcolor{RoyalBlue}{\textsc{highNPQ}})
                    unnatural $\rightarrow$ \textcolor{Plum}{\textsc{outer}}
                    \item V2 ( $=$ \textcolor{WildStrawberry}{\textsc{lowNPQ}}) 
                    $\rightarrow$ \textcolor{Plum}{\textsc{outer}} and
                     \textcolor{Orange}{\textsc{inner}}
                \end{small}
            \end{itemize}       
        \end{column}

        \begin{column}{0.5\textwidth}
            \begin{figure}
                \centering
                \includegraphics[width=\textwidth]{pics/e1_interact_aligned.eps}
                \caption{NPQs means}
                \label{fig:inter5}
            \end{figure}
        \end{column}
    \end{columns}
\end{frame}


\begin{frame}{Discussion: NPQs}
    \begin{columns}
        \begin{column}{0.5\textwidth}
                \begin{itemize}
                    \item[\circled{2}] How does negation interact with evidential bias? 
                \end{itemize}
            \vspace{1em}
            \textsc{strategy}/\textsc{context} interaction
            \begin{itemize}
                \item NPQs are more natural in neutral (unlike in English)
                %\item V2 ( $=$ \textcolor{WildStrawberry}{\textsc{lowNPQ}}) to express negative bias
                \item NCIs $+$ V2 ( $=$ \textcolor{WildStrawberry}{\textsc{lowNPQ}}) $+$ neutral 
                    $\rightarrow$ \textcolor{Plum}{\textsc{outer}}
            \end{itemize}       
        \end{column}

        \begin{column}{0.5\textwidth}
            \begin{figure}
                \centering
                \includegraphics[width=\textwidth]{pics/e1_interact_aligned.eps}
                \caption{NPQs means}
                \label{fig:inter5}
            \end{figure}
        \end{column}
    \end{columns}
\end{frame}

\begin{frame}{Results: PPQs vs NPQs}
    \begin{columns}
        \begin{column}{0.5\textwidth}
            \begin{figure}
                \centering
                \includegraphics[width=\textwidth]{pics/f1_main_aligned.eps}
                \caption{Raw; horizontal line $=$ medians}
                \label{fig:main1}
            \end{figure}
        \end{column}

        \begin{column}{0.5\textwidth}
            \begin{figure}
                \centering
                \includegraphics[width=\textwidth]{pics/f1_interact_aligned.eps}
                \caption{PPQs vs NPQs means}
                \label{fig:inter1}
            \end{figure}
        \end{column}
    \end{columns}
\end{frame}

\begin{frame}{Results: PPQs vs NPQs}
    \begin{columns}
        \onslide<1->{\begin{column}{0.5\textwidth}
            wh-\textit{nibud'} indefinites only
            \vspace{1em}
            \begin{itemize}
                \item[\circled{1}] What is the syntactic-\\semantic 
                status of the negation depending on the question type?
                \begin{itemize}
                    \item V1 \textit{li} no effect/impact of polarity
                    %\item main effect of context 
                \end{itemize}}
                \onslide<2->{\item[\circled{2}] How does negation interact with evidential bias? 
                \begin{itemize}
                    \item \textsc{context/strategy/polarity} interaction
                    \item \textcolor{WildStrawberry}{\textsc{lowNPQs}} unnatural to express positive
                    \item \textcolor{RoyalBlue}{\textsc{highNPQs}} in positive context natural
                \end{itemize}
            \end{itemize}}
        \end{column}

        \onslide<1->{\begin{column}{0.5\textwidth}
            \begin{figure}
                \centering
                \includegraphics[width=\textwidth]{pics/f1_interact_aligned.eps}
                \caption{PPQs vs NPQs means}
                \label{fig:inter3}
            \end{figure}
        \end{column}}
    \end{columns}
\end{frame}

\begin{frame}{Results: \textit{razve}-PQs}
    \begin{columns}
        \begin{column}{0.5\textwidth}
            \begin{figure}
                \centering
                \includegraphics[width=\textwidth]{pics/f2_main_aligned.eps}
                \caption{Raw; horizontal line $=$ medians}
                \label{fig:razve1}
            \end{figure}
        \end{column}

        \begin{column}{0.5\textwidth}
            \begin{figure}
                \centering
                \includegraphics[width=\textwidth]{pics/f2_interact_aligned.eps}
                \caption{\textit{razve}-PQs means}
                \label{fig:razve2}
            \end{figure}
        \end{column}
    \end{columns}
\end{frame}

\begin{frame}{Discussion: \textit{razve}-PQs}
    \begin{columns}
        \begin{column}{0.5\textwidth}
            \begin{itemize}
                \item[\circled{3}] How does the particle \textsc{razve} correlate with evidential bias? 
            \end{itemize}
            \vspace{1em}
            \begin{itemize}
                \begin{small}
                    \item effect of negative and positive contexts 
                    \item acceptable in neutral context due to epistemic bias
                \end{small}
            \end{itemize}
        \end{column}

        \begin{column}{0.5\textwidth}
            \begin{figure}
                \centering
                \includegraphics[width=\textwidth]{pics/f2_interact_aligned.eps}
                \caption{Caption}
                \label{fig:inter1}
            \end{figure}
        \end{column}
    \end{columns}
\end{frame}


\begin{frame}{Cross-Slavic comparison}
    \begin{columns}
        \begin{column}{0.5\textwidth}
            \begin{figure}
                \centering
                \includegraphics[width=\textwidth]{pics/e1_ru_inter_aligned.eps}
                \caption{Russian}
                \label{fig:inter_ru}
            \end{figure}
        \end{column}

        \begin{column}{0.5\textwidth}
            \begin{figure}
                \centering
                \includegraphics[width=\textwidth]{pics/e1_cz_inter_aligned.eps}
                \caption{Czech}
                \label{fig:inter_cz}
            \end{figure}
        \end{column}
    \end{columns}
\end{frame}

\section{Conclusion}
\begin{frame}{Conclusion}
    \begin{itemize}
        \item[\circled{1}] Negation in Russian PQs is primarily interpreted 
        as \textcolor{Plum}{\textsc{outer}} 
        \begin{itemize}
            \item \textcolor{RoyalBlue}{\textsc{high}} $\rightarrow$ \textcolor{Plum}{\textsc{outer}}, 
            \textcolor{WildStrawberry}{\textsc{low}} $\rightarrow$ \textcolor{Plum}{\textsc{outer}} 
            or \textcolor{Orange}{\textsc{inner}}
            \item \textcolor{WildStrawberry}{\textsc{lowNPQs}} with NCIs in netral contexts are interpreted as 
            \textcolor{Plum}{\textsc{outer}}
        \end{itemize}
        \item[\circled{2}] Neutral evidential bias is more natural for negative PQs
        \begin{itemize}
            \item negative bias is not required
            \item positive is also available for \textcolor{RoyalBlue}{\textsc{highNPQs}}
        \end{itemize}
        \item[\circled{3}] Naturalness of \textit{razve}-PQs is affected by evidence
    \end{itemize}
    \pause
    \vspace{1em}
    \begin{block}{Hypothesis}
        The use of negation in Russian (Slavic) PQs might be 
        more closely tied to epistemic than to evidential bias. 
        However, this bias may be weak as compared to the one in English \textsc{highNPQs}. 
    \end{block} 
     
\end{frame}


% \begin{frame}{Conclusion}
%     from Radek's kostra:
%     \begin{itemize}
%         \item Negation in Russian polar questions primarily interpreted as outer negation
%         \item Negation in Russian PQs does not necessarily imply evidential bias
%         \item Razve does imply evidential bias
%         \item What is the role of epistemic bias? \\
%         Hypothesis: the use of negation in Russian (Slavic) polar questions might be 
%         more closely tied to epistemic than to evidential bias, although this bias may 
%         be weak as compared to the one in English (cf. Šimík 2023)
%     \end{itemize}
%     from me: 
%     \begin{itemize}
%         \item INTONATION! cite Esipova and Romero
%     \end{itemize}
% \end{frame}

\begin{frame}[plain]{}
    \begin{center}
        \Large{\textbf{Thank you!}}
    \end{center}
\end{frame}

%\begin{frame}[noframenumbering, plain]{References}
\begin{frame}[allowframebreaks, noframenumbering, plain]{References}
    % Beamer does not support BibTeX so references must be inserted manually as below
    \begin{tiny}
        \bibliography{mybib}
    \end{tiny}
        
\end{frame}



\end{document}